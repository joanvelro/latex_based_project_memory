
%----------------------------------------------------------
%                      Tipo de documento
%-----------------------------------------------------------
\documentclass[11pt,oneside,a4paper]{report}


%--------------------------------------------------
%                  Paquetes Latex
%----------------------------------------------------
\usepackage[latin1]{inputenc}
\usepackage[spanish]{babel}
\usepackage{amsmath}
\usepackage{amsthm}
\usepackage{amsfonts}
\usepackage{amssymb}
\usepackage{float}
\usepackage{graphicx}
\usepackage[pdftex,colorlinks,linkcolor=black,citecolor=black,colorlinks=true,urlcolor=black]{hyperref}
\usepackage{fancybox}
\usepackage{fancyhdr}
\usepackage{anysize} 
\usepackage{lscape} 
\usepackage{array}
\usepackage{pdfpages}
\usepackage{minitoc}
\usepackage{tocloft}





%---------------------------------------------------------------------------------
%                         Definición encabezados
%--------------------------------------------------------------------------------

%   E       EVEN PAGE (PAR)
%   O      ODD PAGE (IMPAR)
%   L       LEFT 
%   C      CENTER
%   R       RIGHT
%    H       HEADER
%    F        FOOTER

\lhead{\includegraphics[scale=0.1]{encabezadoI}}
\rhead{\textsc{\leftmark}}


%\fancyhead[LO]{\includegraphics[scale=0.1]{encabezadoI}}
%\fancyhead[RE]{\textsc{\leftmark}}

%\foot{\textsc{Diseño de una grúa pórtico de exterior}}
%\lfoot{\rightmark} 

%
%\fancyhead[LO,RE]{\chaptermark.} 
%\fancyhead[LO,RE]{\chaptermark.} 
%
%
%\renewcommand{\chaptermark}[1]%
%   {\markboth{\MakeUppercase{\thechapter.\ #1}}{}}
%\renewcommand{\sectionmark}[1]%
%   {\markright{\MakeUppercase{\thesection.\ #1}}}
%\renewcommand{\headrulewidth}{0.5pt}
%\renewcommand{\footrulewidth}{0pt}
%\newcommand{\helv}{%
%   \fontsize{9}{11}\selectfont}
%\fancyhf{}
%\fancyhead[LE,RO]{\includegraphics[scale=0.1]{encabezadoI}}
%\fancyhead[LO]{\rightmark}
%\fancyhead[RE]{ \leftmark}
%\markboth{lllls}{}
%
%
%
%\fancyhead[LO]{\bfseries\rightmark}
%\fancyhead[RE]{\bfseries\leftmark}



%-------------------------------------------------------------------------------------------
%                          Dimensiones del documento
%--------------------------------------------------------------------------------------------
\marginsize{3cm}{3cm}{3cm}{3cm}      % Márgenes
\topmargin -0.5 cm 		                   % Filo superior header - Filo inferior voffset
\evensidemargin 0 cm 	                   % Desfase Eje izquierdo - Texto (pagina par)
\oddsidemargin 0cm  	                   % Desfase Eje izquierdo - Texto (pagina impar)
\textheight 22 cm  		                    % Altura texto
\textwidth 15.5 cm		                     % Anchura texto
\headsep = 1.8cm		                    %Cabecera - Texto
\voffset=-0.2cm		                          	%Cabecera - Borde sup
\raggedbottom




%----------------------------------------------------
%Profundidad del Índice de contenidos
%----------------------------------------------------
\setcounter{tocdepth}{4} 



%-----------------------------------
%Niveles se segmentación
%-----------------------------------
\setcounter{secnumdepth}{5} 




%------------------------------------------------------------------
%               Estilos capitulos
%-----------------------------------------------------------------

%\usepackage[T1]{fontenc}
%\usepackage{titlesec, blindtext, color}
%\definecolor{gray75}{gray}{0.75}
%\newcommand{\hsp}{\hspace{15pt}}
%\titleformat{\chapter}[hang]{\Huge\bfseries}{\thechapter\hsp\textcolor{gray75}{|}\hsp}{0pt}{\Huge\bfseries}



\def\bibname{Bibliografía}


%-----------------------------------------------------------------------------------------------------------------------------------
%                     COMIENZO DOCUMENTO
%-----------------------------------------------------------------------------------------------------------------------------------
\begin{document}


%----------------------------------------
%Listado de ecuaciones
%----------------------------------------
\newcommand{\listequationsname}{List of Equations}
\newlistof{myequations}{equ}{\listequationsname}
\newcommand{\myequations}[1]{%
\addcontentsline{equ}{myequations}{\protect\numberline{\theequation}#1}\par}

%----------------------------------------
%Extensiones 
%----------------------------------------
\DeclareGraphicsExtensions{.jpg,.pdf,.mps,.png,.gif,.fig,.bmp}

%----------------------------------------
%Uso de cabeceras
%----------------------------------------
\pagestyle{fancy}

%----------------------------------------
%Redefinir al español los nombres de las tablas
%----------------------------------------
\renewcommand\listequationsname{Lista de Ecuaciones}
\renewcommand\tablename{Tabla}
\renewcommand\listfigurename{Lista de Figuras}
\renewcommand\listtablename{Lista de Tablas}
\renewcommand\tablename{Tabla}
\renewcommand{\labelitemi}{$\bullet$}




%----------------------------------------
%              PORTADA
%----------------------------------------
\input{portada/portada} 

%----------------------------------------
%     ÍNDICE CONTENIDOS
%----------------------------------------
\thispagestyle{plain}
\cleardoublepage
\thispagestyle{plain}
\dominitoc
\tableofcontents

%----------------------------------------
%        LISTA DE TABLAS
%----------------------------------------
\thispagestyle{plain}
\cleardoublepage
\thispagestyle{plain}
\listoftables

%----------------------------------------
%       LISTA DE FIGURAS
%----------------------------------------
\thispagestyle{plain}
\cleardoublepage
\thispagestyle{plain}
\listoffigures

%----------------------------------------
%      LISTA DE ECUACIONES
%----------------------------------------
%\thispagestyle{plain}
%\cleardoublepage	
%\thispagestyle{plain}								
%\listofmyequations

%----------------------------------------
%        GLOSARIO
%---------------------------------------
\thispagestyle{plain}
\cleardoublepage	
\thispagestyle{plain}
\include{./glosario/glosario}

%----------------------------------------
%        OBSERVACIONES
%----------------------------------------
\thispagestyle{plain}
\cleardoublepage
\include{observaciones/observaciones}






%-----------------------------------------------------------------------------------------------------------------------------
%          MEMORIA
%-----------------------------------------------------------------------------------------------------------------------------
\vspace{0.8cm}
\newpage
\part{Memoria}
\newpage
\cleardoublepage

%-----------------------------------------
%         CAPITULOS
%------------------------------------------
\input{objetivos/objetivos}
\input{clasificacion/clasificacion_grua}
\input{estructura/viga}
\input{estructura/pilar}
\input{estructura/casos}
\input{motores/motores}
%\input{cables/cables}






%--------------------------------------------------------------------------------------------------------------------------------
%    ANEXOS
%------------------------------------------------------------------------------------------------------------------------------
\newpage
\part{Anexo}
\renewcommand\appendixname{Anexo} 
\newpage
\appendix

%----------------------------------------------
%    PARTES DEL ANEXO
%----------------------------------------------
\input{planos/planos}
\input{accesorios/accesorios}
\input{materiales/materiales}
\input{tablas/tablas}
\input{calculos_justificativos/calculos_justificativos}


%-----------------------------------------------------
%Enumeración anexos
%Cambio "Appendice" por "Anexo"
%-------------------------------------------------------
\setcounter{section}{0}										
\renewcommand\thesection{\Alph{subsection}}	





%----------------------------------------------------------------------------------------------------------------------------------
%  BIBLIOGRAFÍA
%----------------------------------------------------------------------------------------------------------------------------------
\newpage
\part{Bibliografía}\label{referencias}
\newpage


\begin{thebibliography} {99}

\bibitem{1} López, M., López, B., Díaz, V. y Fuentes, J. (2012). \emph{Ingeniería del Transporte}. Madrid. Universidad Nacional de Educación a Distancia.
\bibitem{2} UNE 58-112-91/5. Grúas y aparatos de elevación. Clasificación. Parte 5: Grúas puente y pórtico, 1991.
\bibitem{3} UNE 58-113-85. Grúas. Acciones del viento, 1985.
\bibitem{4} UNE 58132-2. Aparatos de elevación. Reglas de cálculo. Parte 2: Solicitaciones y casos de solicitaciones que deben intervenir en el cálculo de las estructuras y de los mecanismos, 2005.
\bibitem{5} Catálogo grúa pórtico Serie SB SHUTTLELIFT. (www.shuttlelift.com)
\bibitem{6} Elementos de hormigón y acero HORMIPRESA. (www.hormipresa.com).
\bibitem{7} UNE 58-112-91/1. Grúas y aparatos de elevación. Clasificación. Parte 1: general, 1991.
\bibitem{8} UNE 58132-2. Aparatos de elevación. Reglas de cálculo. Parte 2: Solicitaciones y casos de solicitaciones que deben intervenir en el cálculo de las estructuras y de los mecanismos, 2005.
\bibitem{9}  UNE 58132-3. Aparatos de elevación. Reglas de cálculo. Parte 3: Cálculo de las estructuras y de las uniones, 2005.
\bibitem{10} Formulario asignatura ``Ingeniería del Transporte''. Curso 2014-2015. Departamento de Ingeniería Mecánica. Universidad Carlos III de Madrid.
\bibitem{11} CESTRI. Juan Tomás Celigüeta. Escuela Superior de Ingenieros. Universidad de Navarra.
www1.ceit.es/asignaturas/estructuras1/Programas.htm
\bibitem{12} Tensión equivalente de Von Misses. www.wikipedia.org
\bibitem{13} Perfiles de acero AcerlorMittal:     www.constructalia.com/espanol/productos/estructuras/tubos/
\bibitem{14} Accesorios para gruas DEMAG: www.demagcranes.es
\end{thebibliography}









\end{document}




